\chapter{Economic Analysis}

Cost was a major concern for the development of the \gls{ceenc}, especially seeing the target audience of hobbyists and students.
Because the project was not funded by any external source, all funding was provided directly by the engineers working on the project.

\section{Market Analysis}
The \gls{cnc} is one of few growing hardware based enterprises in the modern technology market.
In allowing manufacturers greater control over the production process, a wider variety of goods can be produced with limited machinery.
This in turn means that fewer capital resources are required in manufacturing, and production centers can be localized.
Consumers will benefit greatly from the "highly flexible, small-scale manufacturing"\cite{3dprintimpact} provided by \gls{cnc}s.

The \gls{ceenc} brings this technology to an even broader audience, in allowing consumers themselves to produce goods from raw materials.
Persons may circumvent the supply chain entirely, by uploading and milling custom objects on the device.\cite{3dprintsave}
For example, in past years obtaining a differential retainer clip for a 1987 Nissan Hardbody Pickup Truck may have involved a trip to the store, and an order from a centralized warehouse.
This measure could be circumvented entirely however, if the store were to provide \gls{cnc} services, and manufacture the item on the spot. 
Distribution can be made even simpler by making it a household device.

Procuring parts from washing machine load couplers to blender bases with the \gls{ceenc} is as simple as uploading a file.
This in turn increases the overall lifespan of countless products.
While this project has been completed on strictly non-profit grounds, the economic implications for the technology as a whole is staggering.

\section{Manufacturability and Cost Analysis}
Component selection for the \gls{ceenc} was centered around ease of diagnosis, repair, and human assembly.
This in turn puts the system as it stands at a disadvantage for mass production, but avoided several challenging design issues in the project.
Should the device go into production, components will be selected based on size, cost, and package. 
The DRV8825 breakout boards will be placed directly on the device, and as many through-hole components as possible will be replaced with surface mount equivalents.
The \gls{pi} may also be eliminated from the system, and the entire assembly placed on a single board.

While operational environment concerns were largely ignored for the prototyping of this device, its use in high dust conditions delegates the need for conformal coating on all surfaces.
Potting compound should also be applied to reduce damages incurred from excessive vibration.
Finally, the enclosure should be made to manage airflow in a manner that mitigates the migration of mites into the membrane of its heat sinks.

\section{Bill of Materials}
All components required for the design were kept track of in a \gls{bom} to meet the \gls{pcsc} and ensure that the project can be repeatably made by others if need be.
The end unit cost for each of the \gls{ceenc} prototypes as shown in Table ~\ref{table:per-unit-bom} is \$137.08: \$49.58 for the motor driver board, \$32.50 for the motor controller board, and \$55.00 for the \gls{pi} and the enclosure.
The total expenditure for the project as shown in Table ~\ref{table:complete-bom} is \$682.08.