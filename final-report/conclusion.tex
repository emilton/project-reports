\chapter{Conclusion}

Today, \gls{cnc}s are tied to a dedicated computer to control the system.
These systems can be costly and can require a lot of space.
This can make it difficult for students and hobbyists to have access to this type of tool.
\gls{cnc} interfaces have been unintuitive, so the more user-friendly, compact, and affordable \gls{ceenc} will allow for greater availability for students and hobbyists.
This will help these groups decrease the costs and time it takes to prototype a project.

The \gls{ceenc} is composed of three major parts, the user interface, the master controller, and the motor control.
The project was designed using a reiterative engineering process.
Design reviews were used to verify that the project was being built correctly.
During this process, the idea of the \gls{ceenc} was reformulated multiple times in the planning phase before the objectives were decided on and accepted.
From there it was decided to keep the design modular to help with debugging.
The Raspberry Pi was selected to get a powerful device to be the master controller but also at an affordable price that could also host the user interface.
The master controller would take in files and translate the G-code to commands for the motor controller.
The motor controller would then translate those commands to pulses to the motor driver board to control the motors.
All the boards were designed to about the same size as the Raspberry Pi to keep the size of the whole product small.

%\section{Recommendations}
%Provide recommendations, explaining subsequent action or posing specific questions for investigations. 
%One recommendation for the project moving forward would be to looking into a way to consolidate the master controller and motor controller.

%\section{Lessons Learned}
%Discuss the lessons learned. 