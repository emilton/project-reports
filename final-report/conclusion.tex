\chapter{Conclusion}
Restate the problem that gave rise to the report. 
Summarize the main points and the approach that was taken. 

Today, \gls{cnc}s are tied to a dedicated computer to control the system.
These systems can be costly and can require a lot of space.
This can make it difficult for students and hobbyists to have access to this type of tool.
\gls{cnc} interfaces has classically been unintuitive.
Creating a user friendly, compact, and affordable \gls{cnc} interface will allow for greater availability for students and hobbyists.
This will help these groups decrease the costs and time it takes to prototype a project.

The CeeNC is three major parts, the user interface, the master controller, and the motor control.
The project was designed using a reiterative engineering process.
During this process, the idea of the CeeNC was reformulated multiple times in the planning phase before the objectives were decided on and accepted.
From there it was decided to keep the design modular to help with debugging.
The Raspberry Pi was selected to get a powerful device to be the master controller but also at an affordable price.
 
\section{Design Performance}
Summarize the design performance.
 
\section{Recommendations}
Provide recommendations, explaining subsequent action or posing specific questions for investigations. 

\section{Lessons learned}
Discuss the lessons learned. 